\documentclass{article}
\usepackage{enumitem}
\usepackage{setspace}
\title{Physics Summative Exam}
\author{Tristan Simpson}
\begin{document}
\maketitle
\tableofcontents
\doublespacing

\section{Information}
The exam will be worth 238 marks total.
\subsection*{The final exam will be on the following topics:}
\begin{itemize}
    \item Unit 1 - Dynamics and Motion
    \item Unit 2 - Fields
    \item Unit 3 - Momentum + Energy
    \item Unit 4 - Light as a Wave
    \item Unit 5 - Quantum - Bonus
\end{itemize}
\subsection*{The marks will be distributed as follows:}
\begin{itemize}
    \item Unit 1 - Dynamics and Motion (69)
    \item Unit 2 - Fields (30)
    \item Unit 3 - Momentum + Energy (69)
    \item Unit 4 - Light as a Wave (70)
    \item Unit 5 - Quantum (20 Bonus Marks)
\end{itemize}
\subsection*{Mark Distribution Matrix:}
like the one mrs beamer shows in the examples



\section{Personal Notes}
make a seperate and private git repo for this project \\
200 marks total \\
4 units total

% Unit 1 - Dynamics and Motion
\section{Unit 1 - Dynamics and Motion (69)}
\subsection{Solve for each of the following (50)}
Your solutions must include a diagram and a written conclusion to be considered for full marks.

\subsubsection{Incline Plane (10)}
\subsubsection{Elevator (10)}
\subsubsection{Pulleys (10)}
\subsubsection{Solve using Newton's Laws (10)}
\subsubsection{Projectile Motion Type 1 (10)}

\subsection{Theory (5)}
\subsubsection{What are Newton's Three Laws of Motion?}
Your answer should contain a minimum of 100 characters.

\subsubsection{How does uniform motion differ from non-uniform motion?}
Your answer should contain a minimum of 100 characters.

\subsection{Labs (14)}
\subsubsection{What is the procedure for the Projectile Motion Lab? (7)}
Your procedure must be a minimum of 5 steps.
\subsubsection{What is the procedure for the Fletchers Trolley Lab? (7)}
Your procedure must be a minimum of 5 steps.\\

% Unit 2 - Fields
\section{Unit 2 - Fields (30)}
\subsection{Milikans Oil Drop Experiment (10)}
\subsubsection{Draw the diagram for this experiment (5)}
Your diagram must be labeled properly.

\subsubsection{What is the significance of this experiment? (5)}
Your answer should contain a minimum of 300 characters.

\subsection{Solve for each of the following (20)}
Your solutions must include a diagram and a written conclusion to be considered for full marks.

\subsubsection{Electrostatic Forces and Electric Field Intensity (20)}
Three charged objects are located at the vertices of a right triangle. The first charge (Charge A) has a charge of $+6.7\mu C$ and is located at the coordinate $(0, 0)$. The second charge (Charge B) has a charge of $-3.5\mu C$ and is located at the coordinate $(0, 3)$. The third charge (Charge C) has a charge of $-2.0\mu C$ and is located at the coordinate $(4, 0)$. Point D is located at the coordinate $(4, 3)$. The difference in coordinates is measured in centimeters.
\begin{enumerate}[label=\alph*)]
    \item What is the magnitude of the force on Charge A? (5)
    \item What is the magnitude of the force on Charge B? (5)
    \item What is the magnitude of the force on Charge C? (5)
    \item What is the magnitude of intensity on point D? (5)
\end{enumerate}


% Unit 3 - Momentum + Energy
\section{Unit 3 - Momentum + Energy (69)}
\subsection{Theory (5)}
\subsubsection{Descibe how Banked Curves work (3)}
Your answer should contain a minimum of 100 characters.

\subsubsection{What are two ways to reduce the force of a collision? (2)}
Your answer should contain a minimum of 200 characters.

\subsection{Labs (14)}
\subsubsection{What is the procedure for the 2D Momentum Lab? (7)}
Your procedure must be a minimum of 5 steps.

\subsubsection{What is the procedure for the Pith Ball Lab? (7)}
Your procedure must be a minimum of 5 steps.

\subsection{Solve for each of the following (50)}
Your solutions must include a diagram and a written conclusion to be considered for full marks.

\subsubsection{2D Momentum (10)}
\subsubsection{Inelastic Momentum (10)}
\subsubsection{Energy with a Spring (10)}
\subsubsection{Energy + Momentum (10)}
\subsubsection{Impulses (10)}



% Unit 4 - Light as a Wave
\section{Unit 4 - Light as a Wave}
\subsection{Labs (10)}
\subsubsection{What is the procedure for the Young's Double Slit Lab (10)}
Your procedure must be a minimum of 5 steps.

\subsection{Theory (50)}
\subsubsection{Why does an interference pattern appear for single slits? (10)}
Your answer should contain a minimum of 300 characters.

\subsubsection{Why does an interference pattern appear for double slits? (10)}
Your answer should contain a minimum of 300 characters.

\subsubsection{Draw the intensity chart for both double and single slits (10)}
Your intensity charts must be labeled properly.

\subsubsection{Explain how 3D movies work (10)}
Your answer should contain a minimum of 300 characters.

\subsubsection{Briefly summarize each of the following (10)}
Your summaries should contain a minimum of 300 characters.
\begin{enumerate}[label=\alph*)]
    \item Diffraction Gratings
    \item Polarization
    \item Red light vs Green light
\end{enumerate}\leavevmode

\subsection{Solve for each of the following (10)}
Your solutions must include a diagram and a written conclusion to be considered for full marks.

\subsubsection{Double Slit (5)}
Monochromatic light is shone through two slits $4 \times 10^5$ nm apart. The fringes on the screen are $d \times 5 \times 10^3$ away from the slits and have a central maxima width of $2 \times 10^{-4}$ km. What is the wavelength of the light? What color would the light be?

\subsubsection{Single Slit (5)}
How wide is a single slit if it diffracts a 470 nm beam of light such that it produces a central maxima width of double 3.0 cm on a screen 1.2 m away?\\

% Unit 5 - Quantum (20 Bonus Marks)
\section{Unit 5 - Quantum (20 Bonus Marks)}
\subsection{Descibe Wave-Particle Duality (10)}
Your description should contain a minimum of 300 characters.

\subsection{Elaborate on one of the following (10)}
Your description should contain a minimum of 300 characters.
\begin{enumerate}[label=\alph*)]
    \item Schrödinger's Cat
    \item Superposition
    \item Heisenberg Uncertainty Principle
\end{enumerate}\leavevmode\\


% Extra Bonus Marks
\section{Extra Bonus Marks (20)}
\subsection{Momentum + Energy + Kinematics + Forces \#1 (10)}
\subsection{Momentum + Energy + Kinematics + Forces \#2 (10)}


\end{document}