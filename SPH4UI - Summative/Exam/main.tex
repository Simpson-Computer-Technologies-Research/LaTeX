\documentclass{article}
\usepackage{enumitem}
\usepackage{realtristan}
\usepackage{setspace}
\title{Physics Summative Exam}
\author{Tristan Simpson and Oliver Gingerich}
\begin{document}
\maketitle
\tableofcontents
\doublespacing

\section{Mark Distribution $\to$ 120 Marks}
\subsection{Summary}
\begin{itemize}
    \item Unit 1: Dynamics and Motion $\to$ 30 Marks + 10 Bonus
    \item Unit 2: Fields $\to$ 25 Marks
    \item Unit 3: Momentum + Energy $\to$ 30 Marks
    \item Unit 4: Light as a Wave $\to$ 35 Marks
    \item Unit 5: Quantum $\to$ 10 Bonus Marks
\end{itemize}

\subsection{Matrix}
like the one mrs beamer shows in the examples

\subsection{Notes}
\begin{itemize}
    \item Multiple Forces on Bodies: The more $F_{f_{car1}} + F_{f_{car2}}$ the more gas is burned cuz of law of inertia
    \item Fletcher's Trolley: $F_T = F_T + F_f\;(\mu F_g)$
\end{itemize}

% Unit 1 - Dynamics and Motion
\section{Unit 1 - Dynamics and Motion (30)}
\subsection{Incline Plane + Projectile Motion (5)}
A $3.84 \times 10^3$ kg car is parked at the top of a hill $37.8$ m high and $24.6$ m wide inclined at an angle of 64$\degree$.
As the car rolls down the hill, the hill's thick grass creates friction on the cars wheels. The coefficient of
friction ($\mu$) is $7\times 10^{-2}$. After the car reaches the bottom of the hill, it falls
off of a cliff $64.9$ m high with a uniform horizontal velocity.

\vspace{10pt}

\noindent\textbf{Solve for each of the following:}
\begin{enumerate}[label=\alph*)]
    \item How fast is the car travelling by the time it reaches the bottom of the hill?
    \item How far does the car travel (horizontally) after falling off the cliff?
\end{enumerate}\leavevmode

\subsection{Multiple Forces on Bodies (5)}
A 1670kg white truck is pulling a blue vehicle with a mass of 2300kg forward at an angle of
$35\degree$ while a 1590kg green truck is pulling the same vehicle forward at an inclined
angle of $54\degree$. The white and green truck pull the blue vehicle for 2.4 minutes before
running out of gas. The coefficient of friction ($\mu$) is $6.4\times 10^{-1}$.

\vspace{10pt}

\noindent\textbf{Solve for each of the following:}
\begin{enumerate}[label=\alph*)]
    \item What is the force of tension between the vehicles and each rope?
    \item What is the acceleration of the blue vehicle?
    \item What is the velocity of the moving blue vehicle?
    \item Which truck is more likely to run out of gas first?
\end{enumerate}\leavevmode

\subsection{Fletcher's Trolley (5)}
A $2.9 \times 10^3$ g boulder rests on a level plane and is connected to a Fletcher's trolley apparatus
attached to a $6.4$ kg cart suspended in the air. As of now the cart is being held up (not released yet).
The coefficient of friction on the boulder ($\mu$) is $0.54$.

\vspace{10pt}

\noindent\textbf{Solve for each of the following:}
\begin{enumerate}[label=\alph*)]
    \item What is the tension in the string once the cart is released?
    \item What is the acceleration of the trolley?
\end{enumerate}\leavevmode

\subsection{Theory (5)}
\subsubsection{Which of the following are apart of Newton's Laws of Motion?}
\begin{enumerate}[label=\alph*)]
    \item $v = \frac{\Delta d}{\Delta t}$
    \item $F_{net} = ma$
    \item $E_{k} = \frac{1}{2}mv^2$
\end{enumerate}\leavevmode
\subsubsection{Which of the following are apart of Newton's Laws of Motion?}
\begin{enumerate}[label=\alph*)]
    \item For every action there is an equal and opposite reaction.
    \item For every action there is an unequal reaction in the opposite direction.
    \item Earth's gravity causes objects to fall to the ground.
\end{enumerate}

\subsection{Labs (10)}
\subsubsection{What is the procedure for the Projectile Motion Lab? (5)}
Your procedure must be a minimum of 5 steps.

\subsubsection{What is the procedure for the Fletchers Trolley Lab? (5)}
Your procedure must be a minimum of 5 steps.

% Unit 2 - Fields
\section{Unit 2 - Fields (20)}
\subsection{Milikans Oil Drop Experiment (10)}
\subsubsection{Draw the diagram for this experiment (5)}
Your diagram must be labeled properly.

\subsubsection{What is the significance of this experiment? (5)}
Your answer should contain a minimum of 300 characters.

\subsection{Electrostatic Forces and Electric Field Intensity (15)}
Three charged objects are located at the vertices of a right triangle. The first charge (Charge A) has a charge of $+6.7\mu C$ and is located at the coordinate $(0, 0)$. The second charge (Charge B) has a charge of $-3.5\mu C$ and is located at the coordinate $(0, 3)$. The third charge (Charge C) has a charge of $-2.0\mu C$ and is located at the coordinate $(4, 0)$. Point D is located at the coordinate $(4, 3)$. The difference in coordinates is measured in centimeters.
\begin{enumerate}[label=\alph*)]
    \item What is the magnitude of the force on Charge A?
    \item What is the magnitude of the force on Charge B?
    \item What is the magnitude of the force on Charge C?
    \item What is the magnitude of intensity on point D?
\end{enumerate}



% Unit 3 - Momentum + Energy
\section{Unit 3 - Momentum + Energy (30)}
\subsection{Theory (5)}
\subsubsection{Describe how Banked Curves work (3)}
Your answer should contain a minimum of 100 characters.

\subsubsection{What are two ways to reduce the force of a collision? (2)}
Your answer should contain a minimum of 200 characters.

\subsection{Labs (5)}
\subsubsection{What is the procedure for the 2D Momentum Lab? (5)}
Your procedure must be a minimum of 5 steps.

\subsection{Energy + 2D Momentum (10)}

\vspace{5pt}

\begin{minipage}{0.6\textwidth}
    Jeremy rolls a bowling ball with a mass of $6.7$ kg directly at a stationary
    bowling pin that has a mass of $3.0\times 10^{-3}$ kg. The bowling ball is 0.97m above the ground and moving
    $1.2\;m/s$ before Jeremy releases it. After the collision, the bowling ball rolls off at a $32\degree$ angle
    counter-clockwise with a velocity of $3.1\;m/s$. What is the after velocity of the bowling pin?
\end{minipage}
\hspace{5pt}
\begin{minipage}{0.3\textwidth}
    \includegraphics[scale=0.5]{images/bowling.png}
\end{minipage}\leavevmode\\

\subsection{Spring Energy + Inelastic Momentum (10)}

\vspace{5pt}

\begin{minipage}{0.6\textwidth}
    A child; Michael, with a mass of $20$ kg is running towards a rope at a velocity of $2.4\;m/s$. Michael sticks to the rope
    after their collision and swings forward. The rope is very old so when Michael reaches his highest peak, the rope breaks.
    Michael falls to the ground and luckily lands on a trampoline. The trampoline compresses $3.9\;cm$ before bouncing back
    to its original height.
\end{minipage}
\hspace{5pt}
\begin{minipage}{0.3\textwidth}
    \includegraphics[scale=0.35]{images/child_rope.png}
\end{minipage}

\vspace{20pt}

\noindent\textbf{Solve for each of the following:}
\begin{enumerate}[label=\alph*)]
    \item What is the velocity of Michael and the rope after their collision?
    \item What is the velocity of Michael after the collision with the trampoline?
    \item What is the spring constant of the trampoline?
\end{enumerate}\leavevmode

\subsection{Momentum + Energy + Kinematics + Forces (+10)}
On the exoplanet Titan, a stationary boulder rests on the top of a hill $27$ m above the ground.
The gravitational pull on Titan is $7.254$ times less than earth. The boulder slides down the hill at a
velocity of $12\;m/s$ and collides with a stationary spacecraft that has a mass of
$2.0\times 10^4$ kg. The boulder and spacecraft stick together after their collision and
roll together for $13.7$ seconds. The soil at the bottom of the hill has a coefficient of
friction ($\mu$) of $2.4$.

\vspace{10pt}

\noindent\textbf{Solve for each of the following:}
\begin{enumerate}[label=\alph*)]
    \item What is the velocity of the Boulder and Spacecraft after their collision?
    \item How far does the boulder and spacecraft travel? ($\Delta d$)
    \item How does being on a different planet affect the variables in this equation?
\end{enumerate}\leavevmode

% Unit 4 - Light as a Wave
\section{Unit 4 - Light as a Wave (35)}
\subsection{Theory (25)}
\subsubsection{Why does an interference pattern appear for single slits? (5)}
Your answer should contain a minimum of 200 characters.

\subsubsection{Why does an interference pattern appear for double slits? (5)}
Your answer should contain a minimum of 200 characters.

\subsubsection{Draw the intensity chart for both double and single slits (5)}
Your intensity charts must be labeled properly.

\subsubsection{Briefly summarize each of the following (10)}
Your summaries should contain a minimum of 120 characters.
\begin{enumerate}[label=\alph*)]
    \item Diffraction Gratings
    \item Polarization
    \item Red light vs Green light
\end{enumerate}\leavevmode

\subsection{Double Slit (5)}
Monochromatic light is shone through two slits $4 \times 10^5$ nm apart. The fringes on the screen are $d \times 5 \times 10^3$ away from the slits and have a central maxima width of $2 \times 10^{-4}$ km. What is the wavelength of the light? What color would the light be?

\subsection{Single Slit (5)}
How wide is a single slit if it diffracts a 470 nm beam of light such that it produces a central maxima width of double 3.0 cm on a screen 1.2 m away?\\

% Unit 5 - Quantum (+10 Bonus Marks)
\section{Unit 5 - Quantum (+10 Bonus)}
\subsection{Describe Wave-Particle Duality (+5)}
Your description should contain a minimum of 300 characters.

\subsection{Elaborate on one of the following (+5)}
Your description should contain a minimum of 300 characters.
\begin{enumerate}[label=\alph*)]
    \item Schrödinger's Cat
    \item Superposition
    \item Heisenberg Uncertainty Principle
\end{enumerate}\leavevmode\\


\end{document}