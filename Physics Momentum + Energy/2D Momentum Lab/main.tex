% // Packages and Document Class
\documentclass{article}
\usepackage{multirow}
\usepackage{cancel}
\usepackage{amssymb}

% // Title
\title{2D Momentum Lab}
\author{SPH4UI - Tristan, Oliver, + 2}
\begin{document}
\maketitle

% // The purpose of this lab. What did you do?
\section*{Purpose}
The purpose of this lab was to determine whether the law of conservation of momentum is present between two colliding marbles of equal mass.
To determine whether momentum was conserved, we used the formula: $P_{tot} = P_{tot}\prime$ along with components in both $x$ and $y$ directions.


% // The materials that were used in the lab
\section*{Materials}
\begin{enumerate}
    \item {\textbf{One} metal Ramp.}
    \item {\textbf{One} type C Clamp.}
    \item {\textbf{Two} metal marbles of equal mass.}
    \item {\textbf{One} meter stick.}
    \item {\textbf{Three} sheets of regular blank paper.}
    \item {\textbf{Three} sheets of carbon paper.}
\end{enumerate}


% // The steps your group took to perform the lab successfully
\section*{Procedures}
\textbf{Part 1:}
\begin{enumerate}
    \item {The metal ramp was clamped to the edge of the desk.}
    \item {Both the regular paper and carbon paper were positioned on the ground, straight infront of the ramp.}
    \item {The two papers were taped to fortify their position, with the carbon paper ontop of the regular paper.}
    \item {The marble was dropped from the ramp, with it hitting the paper below.}
    \item {The distance from the table to this point was noted.}
    \item {The above steps were repeated another \textbf{five} times.}
\end{enumerate}\leavevmode\newline
\textbf{Part 2:}
\begin{enumerate}
    \item {The other sheets of carbon and regular paper were placed to the left and right sides of the first sheet.}
    \item {These other sheets were positioned for the colliding marbles to hit them.}
    \item {The ramp was angled slightly to the right.}
    \item {The second marble was placed at the front of the ramp on a screw.}
    \item {The first marble was released.}
    \item {After the marbles collided and hit the ground paper, the distance between the desk and the points was noted. (both $x$ and $y$)}
    \item {The above steps were repeated another \textbf{five} times.}
\end{enumerate}\leavevmode


% // Observations for each marble
\section*{Observations}
Data provided by Halle Nunes\newline\newline
\begin{tabular}{ |c|c|c|c|c| }
    \hline
    Marble & $v_{x}$ & $v_{y}$  & $v_{xc}$ & $v_{yc}$ \\
    \hline
    1      & 0m/s    & 43.15m/s & 15.6m/s  & 22.1m/s  \\
    \hline
    2      & 0m/s    & 0m/s     & 19.7m/s  & 23.6m/s  \\
    \hline
\end{tabular}.\newline\newline


% // Calculating Ptotx and Ptotx Prime
\section*{Calculations ($P_{tot_{x}}$)}
$P_{tot_{x}} = P_{tot_{x}}\prime$
\newline\newline
$m_{x_{1}}v_{x_{1}} + m_{x_{2}}v_{x_{2}} = m_{xc_{1}}v_{xc_{1}} + m_{xc_{2}}v_{xc_{2}}$
\newline\newline
$(\cancel{m}_{x_{1}}) (v_{x_{1}}) + (\cancel{m}_{x_{2}})(v_{x_{2}})
    = (\cancel{m}_{xc_{1}})(v_{xc_{1}}) + (\cancel{m}_{xc_{2}})(v_{xc_{2}})$
\newline\newline
$\cancel{v}_{x_{1}}^{0} + \cancel{v}_{x_{2}}^{0} = (-15.6m/s) + (19.7m/s)$
\newline\newline
$0 = 4.1000m/s$
\newline\newline
$\therefore P_{tot_{x}} = 0$
\newline
$\therefore P_{tot_{xc}} = 4.1000m/s$


% // Calculating Ptoty and Ptoty Prime
\section*{Calculations ($P_{tot_{y}}$)}
$P_{tot_{y}} = P_{tot_{y}}\prime$
\newline\newline
$m_{y_{1}}v_{y_{1}} + m_{y_{2}}v_{y_{2}} = m_{yc_{1}}v_{yc_{1}} + m_{yc_{2}}v_{yc_{2}}$
\newline\newline
$(\cancel{m}_{y_{1}}) (v_{y_{1}}) + (\cancel{m}_{y_{2}})(v_{y_{2}})
    = (\cancel{m}_{yc_{1}})(v_{yc_{1}}) + (\cancel{m}_{yc_{2}})(v_{yc_{2}})$
\newline\newline
$(43.15m/s) + \cancel{v}_{y_{2}}^{0} = (22.1m/s) + (23.6m/s)$
\newline\newline
$43.150 = 45.700m/s$
\newline\newline
$\therefore P_{tot_{y}} = 43.150m/s$
\newline
$\therefore P_{tot_{yc}} = 45.700m/s$
\newline

% // Calculating Ptot and Ptot Prime
\section*{Calculations ($P_{tot}$) and ($P_{tot}\prime$)}

% // Ptot Normal
$P_{tot} = \sqrt[]{{(P_{tot_{x}})}^{2} + {(P_{tot_{y}})}^{2}}$
\newline\newline
$P_{tot} = \sqrt[]{{(0)}^{2} + {(43.150)}^{2}}$
\newline\newline
$P_{tot} = 43.150\frac{kgm}{s}$
\newline\newline\newline
$P_{tot}\prime = \sqrt[]{{(P_{tot_{xc}})}^{2} + {(P_{tot_{yc}})}^{2}}$
\newline\newline
$P_{tot}\prime = \sqrt[]{{(4.1000)}^{2} + {(45.700)}^{2}}$
\newline\newline
$P_{tot}\prime = 45.883\frac{kgm}{s}$
\newline\newline\newline
$\therefore$ Since $P_{tot} \neq P_{tot}\prime$ momentum was not conserved.
\newline

\section*{Sources of Error}
\subsection*{Errors}
\begin{enumerate}
    \item {The table that the ramp was on had an extending edge. This made measuring the distance between the table and the points harder and potentially inaccurate.}
    \item {The ramp was rotated too much.}
    \item {The marble's masses were not the same.}
\end{enumerate}
\subsection*{Solutions}
\begin{enumerate}
    \item {Place the ramp on a table with no edge.}
    \item {Using a protractor to measure the distance the ramp was rotated would resolve this error}
    \item {Measuring the weight of the marbles to ensure that they are the same mass would resolve this error.}
\end{enumerate}\leavevmode

\section*{Conclusion}
This lab efficiently demonstrated the law of conservation of momentum between two colliding objects.
By the data provided through our group experiments, it was resolved that the law of
conservation of momentum was \textbf{\textit{not}} present.
\newline

\section*{Synthesis}
The learnings from this lab are applicable to numerous real-world scenarios.
The first application is a game of pool. In a game of pool, the cue ball is shot at other at-rest balls on the billiards table.
Secondly, having a direct correlation to this lab are car collisions. The law of conservation of momentum is consistently utilized
throughout the tests of a cars safety and strength. Finally, this lab can be applied to moving pendulums.

\end{document}