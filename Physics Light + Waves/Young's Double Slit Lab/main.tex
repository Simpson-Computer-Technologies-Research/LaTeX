\documentclass{article}
\usepackage{multirow}
\usepackage{changepage}
\usepackage{graphicx}
\usepackage[fleqn]{amsmath}
\usepackage{amssymb}
\title{Young's Double Slit Lab}
\author{Tristan, Ibrahim, Oliver, Aaliya}
\begin{document}
\maketitle
\section*{Purpose} {
    The purpose of this lab was to measure the wave length ($\lambda$) of a laser travelling through a slit, then calculate the error percentage produced by our result.
}
\section*{Materials} {
    \begin{itemize}
        \item {\textbf{One} Slit}
        \item {\textbf{One} Line of Measuring Tape}
        \item {\textbf{One} Red Laser}
        \item {\textbf{One} Projector}
        \item {\textbf{Two} Rulers}
        \item {\textbf{One} Piece of Paper}
    \end{itemize}\leavevmode
}
\section*{Procedure} {
    \begin{enumerate}
        \item {The sheet of paper was taped to the black-board.}
        \item {The laser was positioned infront of the slit, both facing the piece of paper on the black-board.}
        \item {Using a pencil, the blocks made by the laser and slit were marked on the piece of paper.}
        \item {The distance between the middles of two blocks made by the laser and slit was measured. ($\Delta x$)}
        \item {The distance from the slit to the black-board was measured. ($L$)}
        \item {The width of the slits' double-line was measured using the projector, two rulers, and the slit itself. ($d$)}
    \end{enumerate}\leavevmode
}
\section*{Observations} {
    During our experiment, our group made the following observations. Below exhibits the distance from the slit to the black-board ($L$) and the distance from the middles of two blocks projected by the laser travelling through our slit. ($\Delta x$)\newline\newline
    $\to\;\;L$ $\approx$ $73cm$ $\approx$ $7.3m$\newline\newline
    $\to\;\;\Delta x$ $\approx$ $0.7cm$ $\approx$ $7.0\times 10^{-3}$ $m$\newline\newline
    $\to\;\;Theoretical\;\lambda \approx 6.53 \times 10^{-7}m$\newline
}
\section*{Calculations: Solve for $d$} {
    Before solving for $\lambda$, we must find $d$ (the width of the slit). To do this we used a projector (magnifier) and the formula below.\newline\newline
    \begin{equation*}
        \therefore\;d\;=\;\frac{(1mm)(slit\;width\;on\;projection)}{(1mm\;on\;projection)}\;=\;\frac{(1mm)(3.2mm)}{(6.1mm)}\;\approx\;5.24\times\;10^{-4}m
    \end{equation*}\leavevmode\newline
}
\section*{Calculations: Solve for $\lambda$} {
    After finding $d$ we can solve for $\lambda$ by substituting our previous variables into the following equation.\newline\newline
    \begin{equation*}
        \therefore\;\lambda\;=\;\frac{\Delta\;xd}{L}\;=\;\frac{(7.0\times\;10^{-3})(5.24\times\;10^{-4})}{(7.3)}\;\approx\;5.02\;\times\;10^{-7}m
    \end{equation*}\leavevmode\newline
}
\section*{Calculations: Error Percentage} {
    After solving for our experimental result, we use the error percentage formula to find how accurate our results really were.
    \begin{equation*}
        \therefore\;Error\;\%\;=\;\left(\frac{Experimental\;-\;Theoretical}{Theoretical}\right)\;\times\;100
    \end{equation*}\leavevmode
    \begin{equation*}
        \therefore\;Error\;\%\;=\;\left(\frac{(5.02\;\times\;10^{-7})\;-\;(6.53\;\times\;10^{-7})}{(6.53\;\times\;10^{-7})}\right)\;\times\;100\;\approx\; 22.97\;\%
    \end{equation*}
}
\section*{Sources of Errors} {
    \begin{enumerate}
        \item {The blocks projected by the laser travelling through the slit were too close together. This made measuring the distances far harder and far more inaccurate.}
        \item {Measuring from the projector made achieving accurate results far more difficult due to how large and pixelated the projection is.}
        \item {The laser was too far away / too close to the the slit.}
    \end{enumerate}
}
\section*{Solutions to Errors} {
    \begin{enumerate}
        \item {Move to a larger space so we can station the materials accordingly.}
        \item {Use a higher resolution projector or have each of our group members measure values then calculate the average of the results.}
        \item {Measure an appropriate distance for the laser away from the slit.}
    \end{enumerate}\leavevmode
}
\section*{Questions} {
    \textbf{1. How would changing the laser light from red to green affect the different variables of this experiment?}\newline {
        Changing the laser light from red to green would affect the different variables of this experiment because the color of the wave is dependant on it's wavelength ($\lambda$). Thus, since the red laser light has a longer wavelength than the green laser light, our $\lambda$ is much larger.\newline\newline
    }
    \textbf{2. How would increasing the separation of the slits affect the different variable of this experiment?}\newline {
        Increasing the seperation of the slits ($d$) affects the different variables of this experiment since light waves must overlap to have interference. This means having too large of a slit would not produce enough overlap and too small of a slit would not let enough light through. This directly affects the distance between the middles of blocks ($\Delta x$) projected by the laser light.
    }
}
\section*{Conclusion} {
    By the completion of this lab, it was determined that the wave length ($\lambda$) of a laser travelling through a slit was approximately $5.02\;\times 10^{-7}m$ with a calculated error percentage of approximately $22.97\;\%$.\newline\newline Since our experiment had utilized a red laser, our resulting wave length ($\lambda$) should have ranged between $\approx 6.2 \times 10^{-7} < \lambda < 7.5 \times 10^{-7}\;m$
}
\end{document}