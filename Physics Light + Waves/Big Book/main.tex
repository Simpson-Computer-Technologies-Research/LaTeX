\documentclass{article}
\usepackage[]{realtristan}
\title{Physics Light and Waves Summary}
\author{Tristan Simpson}
\begin{document}
\maketitle


\header{Bubbles}


\header{Young's Double Slit}


\header{Diffraction Grading}


\header{Polarization}

% Theory Questions
\section{Theoretical Question 1}
\subsection*{Question}
If Young's Double Slit Experiment were submerged in water, how would the fringe pattern change?
\subsection*{Answer}
The length of the wave is strongly dependant on it's medium. If the experiment were submerged in water and not air (two different mediums), the wavelength of the light would decrease, and the fringe pattern would become closer together.\\

\section{Theoretical Question 2}
\subsection*{Question}
For diffraction by a single slit, what is the effect of increasing (a) the slit width, and (b) the wavelength?
\subsection*{Answer}
(a) Increasing the slit width ($d$) will decrease the width of the central maxima because of the equation $\Delta y = \left(\frac{(\lambda)(L)}{d}\right)$\\\\
(b) Increasing the wavelength ($\lambda$) will increase the sie of the central maxima because of the equation $\Delta y = \left(\frac{(\lambda)(L)}{d}\right)$\\

\section{Theoretical Question 3}
\subsection*{Question}
How can you tell if a pair of sunglasses is polarizing or not?
\subsection*{Answer}
The easiest way to tell whether a pair of sunglasses is polarized or not is by closing either one of your eyes and looking at another person wearing polarized sunglasses. You will notice that one of the other person's lenses blocks all light from going through. This happens because only either the vertical or horizontal (x or y) components of the light wave can travel through each lense. Because of this, you can't see through the other person's lense with opposite components of your open-eye's lense.  (Vertical waves cannot travel through a horizontal polarized lense, and vice versa.)\\



% Double Slit Equation 1
\section{Double Slit Equations (1)}
Monochromatic light falls on two slits 0.024 mm apart. The fringes on a screen 3.00 m away are 6.7 cm apart. What is the wavelength of light?
\subsection*{Givens}
\begin{itemize}
    \item $\Delta x = 2.4 \times 20^{-5}\;m$
    \item $L = 3.00\;m $
    \item $d = 6.7 \times 10^{-2}\;m$
\end{itemize}\leavevmode

\subsection*{Solve}
To solve for the wavelength of the light, we can use the equation $\lambda = \frac{(\Delta x)(d)}{L}$. Our result should be in meters, but if converted, should be in nanometers.\\
\begin{align*}
     & \lambda = \left(\frac{(\Delta x)(d)}{L}\right)      = \left(\frac{(6.7 \times 10^{-2})(2.4 \times 10^{-5})}{3.00}\right) \approx 5.20 \times 10^{-7}\;m
\end{align*}\leavevmode\\\\\\\\

% Double Slit Equation 2
\section{Double Slit Equations (2)}
A parallel beam of 700 nm light falls on two small slits $6.0 \times 10^{-2}$ mm apart. How wide would a pattern of eight bright fringes be on a screen 3.0 m away?
\subsection*{Givens}
\begin{itemize}
    \item $d = 6.0 \times 10^{-5}\;m$
    \item $\lambda = 7.0 \times 10^{-7}\;m$
    \item $L = 3.0\;m$
\end{itemize}\leavevmode

\subsection*{Solve}
To solve for the width of the pattern of eight bright fringes, we must first solve for the distance between two fringes ($\Delta x$). After solving for $\Delta x$ we can multiply it's value by eight to get the final distance.\\
\begin{align*}
     & \Delta x = \frac{(\lambda)(L)}{d} = \left(\frac{(7.0 \times 10^{-7})(3.0)}{6.0 \times 10^{-5}}\right) \approx 3.5 \times 10^{-2}\;m \\\\
     & \therefore\;\;(8)\Delta x = (8)(3.5 \times 10^{-2}) \approx 2.8 \times 10^{-1}
\end{align*}\leavevmode\\



% Single Slit Equations
\section{Single Slit Equations (1)}
How wide is the central diffraction peak on a screen 2.50 m behind a 0.0212 mm wide slit illuminated by 550 nm light?
\subsection*{Givens}
\begin{itemize}
    \item $L = 2.50\;m$
    \item $d = 2.12 \times 10^{-5}\;m$
    \item $\lambda = 5.5 \times 10^{-7}\;m$
\end{itemize}\leavevmode
\subsection*{Solve}
To solve for the width of the central maxima we use an equation very similar to the one used in the double slit calculations: $\Delta y = \left(\frac{(\lambda)(L)}{d}\right)$ Where $\Delta y$ is the width of the central maxima.\\
\begin{align*}
     & \therefore\;\;\Delta y = \left(\frac{(\lambda)(L)}{d}\right) = \left(\frac{(5.5 \times 10^{-7})(2.50)}{2.12 \times 10^{-5}}\right) \approx 6.5 \times 10^{-2}\;m
\end{align*}\leavevmode\\

% Single Slit Equation 2
\section{Single Slit Equations (2)}
How wide is a slit if it diffracts 690 nm light so that its central peak is 3.0 cm wide on a screen 2.80 m away?
\subsection*{Givens}
\begin{itemize}
    \item $L = 2.80\;m$
    \item $ \Delta y = 3.0 \times 10^{-2}\;m$
    \item $\lambda = 6.9 \times 10^{-7}\;m$
\end{itemize}\leavevmode
\subsection*{Solve}
To solve for the width of the single slit we can rearrange the formula from the solve above. $\Delta y = \left(\frac{(\lambda)(L)}{d}\right)$ $\to$ $d = \left(\frac{(\lambda)(L)}{\Delta y}\right)$.\\
\begin{align*}
     & \therefore\;\;d = \left(\frac{(\lambda)(L)}{\Delta y}\right) = \left(\frac{(6.9 \times 10^{-7})(2.80)}{3.0 \times 10^{-2}}\right) \approx 6.44 \times 10^{-5}\;m
\end{align*}\leavevmode\\

\end{document}

\begin{adjustwidth}{1cm}{0pt}
    \subsection{Young's Double Slit (1)}
    \begin{adjustwidth}{0.5cm}{0pt}
        A sodium vapour light illuminates with monochromatic yellow light two narrow slits that are 1.0mm apart.
        If the viewing screen is 1.0m away from the slits and the distance from the central bright line to the
        next bright line is 0.589mm, what is the wavelength of the light?\newline

        \noindent\textbf{Givens}
        \begin{itemize}
            \item $\Delta x = 0.589mm = 5.89 \times 10^{-4}m$
            \item $d = 1.0mm = 1.0 \times 10^{-3}m$
            \item $L = 1.0m$
        \end{itemize}\leavevmode\newline

        \noindent\textbf{Solving for real $\Delta x$}\\
        To solve for $\Delta x$ we must first take a look at the question above.
        Recognize the statement: "the central bright line to the next bright line".
        Because this question specifies that the distance is exactly one line in distance, we set the $\Delta x\;multiplier$ to 1

        \lalign {
            show\;diagram\;here
        }
        \lalign{
            \Delta x &= (\Delta x multiplier)(\Delta x) \\
            &= (1)(5.89 \times 10^{-4}) \\
            &= 5.89 \times 10^{-4}
        }\newline

        \noindent\textbf{Solving for Lambda}
        \lalign{
            \lambda = \frac{(\Delta x)(d)}{L}
        }
    \end{adjustwidth}
\end{adjustwidth}
\end{document}