\documentclass{article}
\usepackage[]{realtristan}
\title{Physics Light and Waves Summary}
\author{Tristan Simpson}
\begin{document}
\maketitle


\header{Bubbles}


\header{Young's Double Slit}


\header{Diffraction Grading}


\header{Polarization}

\header {Example Equations}
\begin{adjustwidth}{1cm}{0pt}
    \subsection{Young's Double Slit (1)}
    \begin{adjustwidth}{0.5cm}{0pt}
        A sodium vapour light illuminates with monochromatic yellow light two narrow slits that are 1.0mm apart.
        If the viewing screen is 1.0m away from the slits and the distance from the central bright line to the
        next bright line is 0.589mm, what is the wavelength of the light?\newline

        \noindent\textbf{Givens}
        \begin{itemize}
            \item $\Delta x = 0.589mm = 5.89 \times 10^{-4}m$
            \item $d = 1.0mm = 1.0 \times 10^{-3}m$
            \item $L = 1.0m$
        \end{itemize}\leavevmode\newline

        \noindent\textbf{Solving for real $\Delta x$}\\
        To solve for $\Delta x$ we must first take a look at the question above.
        Recognize the statement: "the central bright line to the next bright line".
        Because this question specifies that the distance is exactly one line in distance, we set the $\Delta x\;multiplier$ to 1

        \lalign {
            show\;diagram\;here
        }
        \lalign{
            \Delta x &= (\Delta x multiplier)(\Delta x) \\
            &= (1)(5.89 \times 10^{-4}) \\
            &= 5.89 \times 10^{-4}
        }\newline

        \noindent\textbf{Solving for Lambda}
        \lalign{
            \lambda = \frac{(\Delta x)(d)}{L}
        }
    \end{adjustwidth}
\end{adjustwidth}
\end{document}