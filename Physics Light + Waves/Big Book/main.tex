\documentclass{article}
\usepackage[]{realtristan}
\title{Physics Light and Waves Summary}
\author{Tristan Simpson}
\begin{document}
\maketitle
\tableofcontents

\vspace{3cm}
\section{Bubbles and Films}
\subsection{Fixed and Free Ends}
When a wave hits a fixed end, the reflecting wave (phase) is inverted. When a wave hits a free end, the phase is originially oriented. When a wave hits either a fixed or free end, the wave does NOT get transmitted (no wave on the other side of the fixed/free end).
A wave interacting with a Fixed or Free End is not the same as a wave travelling from a Fast to Slow/Slow to Fast medium, though they are similar.\\
\includegraphics[scale=3]{images/fixed_free_ends}

\vspace{-2cm}

\subsection{Mediums}
So what is a medium? A medium is the material that the wave is travelling through. A great example of mediums, directly correlating to this unit, are air and water. Air is a fast medium, water is a slow medium, and the sunlight is the wave.\\\\
\textbf{Fast to Slow Mediums}\\
The transmitted wave is ALWAYS originally oriented!
\begin{itemize}
    \item Reflected wave is inverted
    \item Transmitted wave is originally oriented
    \item The amplitudes of transmitted and reflected wave are smaller than the original wave. Though if both their amplitudes were added together, they would equal the amplitude of the original wave.
\end{itemize}\leavevmode

\vspace{1pt}

\begin{adjustwidth}{1cm}{0pt}
    \includegraphics[scale=0.4]{images/fast_slow_wave}
\end{adjustwidth}\leavevmode

\vspace{2cm}

\noindent\textbf{Slow to Fast Mediums}\\
The transmitted wave is ALWAYS originally oriented!
\begin{itemize}
    \item Reflected wave is originally oriented
    \item Transmitted wave is originally oriented
    \item The amplitudes of transmitted and reflected wave are smaller than the original wave. Though if both their amplitudes were added together, they would equal the amplitude of the original wave.
\end{itemize}\leavevmode

\vspace{1pt}

\begin{adjustwidth}{1cm}{0pt}
    \includegraphics[scale=0.35]{images/slow_fast_wave}
\end{adjustwidth}\leavevmode

\subsection{Bubbles as a Film}
As the light wave travels from the air medium ($n_{air}$) to the soap medium ($n_{soap}$), the phase is reflected back to the persons eye and the original wave is inverted into the bubble. The original wave is then reflected back to the persons eye after bouncing around inside the bubble. The person now see's two reflected lights. Depending on the thickness of the film ($t$), the reflected light ($R$) and the original light ($O$) will either constructively or deconstructively interfere.\\\\
\includegraphics[scale=0.05, angle=90, origin=c]{images/bubble_lens}

\vspace{-1.5cm}

\subsection{Bubble Colors}
In a bubble you see three primary colors: cyan, magenta, and yellow. Long wavelengths (eg. red laser light) need a thicker bubble wall to get out of step than short wavelengths (eg. violet laser light). When red is cancelled, it leaves a blue-green reflection. As the bubble thins, you see magenta when green is cancelled out, you see cyan when yellow is cancelled out, and you see yellow when blue is cancelled out.\\\\

\subsection{Reflected Film Light}
If a second wave is out of sync with the original wave, the two waves will either deconstructively or constructively interfere by the time they reach a screen (eye, screen, table, etc.) which will result in seeing either a bright or dark spot.\\\\
If the thickness of a film is very, very thin, (practically zero), then the waves will be in sync because the distance is negligible. \\\\
If the thickness of a film is thicker, then the waves, depending on the thickness of the film, will potentially be out of sync. \\\\
This is because $Wave_2$ has to travel farther than $Wave_1$ post it's reflection on the bottom of the film, to reach the screen.\\\\
\includegraphics[scale=0.5]{images/reflected_films} \\\\
\noindent \textbf{Reflection Constructive Interference (Bright):} $\frac{\lambda}{4}$, $\frac{3\lambda}{4}$, $\frac{\lambda}{4}$ \\\\
\textbf{Reflection Deconstructive Interference (Dark):} $0$, $\frac{\lambda}{2}$, $\lambda$, $\frac{3\lambda}{2}$

\subsection{Transmitted Film Light}
The transmitted film light constructive and deconstructive interference equations are the exact opposite of the reflected film light constructive and deconstructive interference equations.\\\\
\includegraphics[scale=0.5]{images/transmitted_films} \\\\
\noindent \textbf{Reflection Constructive Interference (Bright):} $0$, $\frac{\lambda}{2}$, $\lambda$, $\frac{3\lambda}{2}$ \\\\
\textbf{Reflection Deconstructive Interference (Dark):} $\frac{\lambda}{4}$, $\frac{3\lambda}{4}$, $\frac{\lambda}{4}$ \\

\section{Single and Double Slits}
\subsection{Theory}
\textbf{If Young's Double Slit Experiment were to be submerged in water, how would the fringe pattern change?}\\\\
The length of a wave ($\lambda$) is strongly dependant on it's medium. If the experiment was to be submerged in water (a slower medium than air), then the wavelength of the light would decrease and the fringe pattern would become closer together.\\\\

\noindent\textbf{2. For diffraction by a single slit, what is the effect of increasing (a) the slit width, and (b) the wavelength?}\\\\
(a) Increasing the slit width ($d$) would decrease the width of the central maxima. This is proven by the equation: $\Delta y = \left(\frac{(\lambda)(L)}{d}\right)$\\\\
(b) Increasing the wavelength ($\lambda$) would increase the sie of the central maxima. This is proven by the equation: $\Delta y = \left(\frac{(\lambda)(L)}{d}\right)$\\\\

\noindent\textbf{3. What makes light waves colourful?}\\\\
The colour of light is determined by it's wavelength. The longer the wavelength, the more red the light is. The shorter the wavelength, the more blue the light is. This is why green laser lights have a shorter wavelengths compared to red laser lights.\\\\

\noindent\textbf{4. How did the Double Slit Experiment prove light was a wave?}\\\\
Isaac Newton believed that light was a particle but Christiaan Huygens believed that light was a wave. Sadly at the time, Christiaan Huygens didn't have an effective experiment to test whether light was a wave because monochromatic light (artifical light) had not been discovered. In the year 1801, Thomas Young invented the Double Slit Experiment which proved that light was a wave. If light was a particle there would be no interference pattern, instead there woul only be two lines made from the two slits.

\vspace{3cm}

\subsection{Wavelets}
Christiaan Huygens use of "wavelets" describes the interference pattern created by a single slit. Wavelets are a wave split into multiple miniatures lines and depending on the $\theta$ of these lines, the screen (or eye, etc.) will experience either constructive or deconstructive interference.\\

\noindent\textbf{Deconstructive Interference}\\
If the path difference from one side of the slit to the other is a multiple of $\lambda$ (eg. $2\lambda$, $3\lambda$, ...), then there will be deconstructive interference.
\begin{align*}
    \therefore y_m = \left(\frac{m\lambda L}{w}\right)\;\; where\;\;(m\;=\;\pm 1,\;\pm 2,\;\pm 3,\;\pm 4,\;...)
\end{align*}\leavevmode\\

\noindent\textbf{Constructive Interference}\\
If the path difference from one side of the slit to the other is off by half a wave length (eg. $\lambda + \frac{1}{2}$, $\lambda - \frac{1}{2}$, ...), then there will be constructive interference.
\begin{align*}
    \therefore y_m = \left(\frac{(m + \frac{1}{2})\lambda L}{w}\right)\;\; where\;\;(m\;=\;\pm 1,\;\pm 2,\;\pm 3,\;\pm 4,\;...)
\end{align*}\leavevmode
\includegraphics[scale=0.8]{images/wavelets} \\

\subsection{Patterns}
The patterns below are the result of a single slit and a double slit. The intensity's of the fringes are also shown. The highest points of the intensities are called the maximas, and the lowest points of the intensities are called the minimas.\\\\
\begin{minipage}{0.5\textwidth}
    \textbf{Double Slit Pattern $\Delta x = \frac{(\lambda)(d)}{L}$} \\
    \includegraphics[scale=0.5]{images/double_slit_pattern} \\
\end{minipage}
\begin{minipage}{0.5\textwidth}
    \vspace{10pt}
    \textbf{Single Slit Pattern $\Delta y = \frac{(\lambda)(d)}{L}$} \\
    \includegraphics[scale=0.5]{images/single_slit_pattern} \\\\
\end{minipage}\leavevmode

\subsection{Diffraction}
When a light wave passes through a narrow slit, the wave diffracts. This diffraction is only noticeable by the human eye when $\frac{\lambda}{d} \geq 1$ \\\\
The interference pattern created by the single slit is from wavelets. The interference pattern created by the double slit is from the waves overlapping eachother, causing constructive and deconstructive interference. \\\\

\section{Diffraction Gratings}
Young's Double Slit apparatus inspired the diffraction grating. The diffraction grating is a series of NUMEROUS slits (not just the two from young's "double" slit) that are very close together.\\
\begin{itemize}
    \item Diffraction Gratings gave physisists a much more accurate measurement of the wavelength of the light source.
    \item The name "Diffraction Grating" comes from the fact that the light waves diffract when as they travel through the tiny slits cut into the apparatus.
    \item Holding a diffraction grating to a source of \textbf{white light} easily splits the light into a rainbow of colors.
    \item The patterns created by a diffraction grating still consist of maximas (bright regions / high intensities) and minimas (dark regions / low intensities) except the maximas become very narrow and well defined.
    \item Using monochromatic light you can see very distinct bright and dark fringes.
\end{itemize}\leavevmode\\
\noindent\textbf{Image of Diffraction Grating:}
The light doesn't travel through the black lines. The light travels through the white lines. These lines act as mini slits.\\\\
\includegraphics[scale=0.4]{images/diffraction_grating} \\

\section{Polarization}
Polarization is the splitting of waves into their components.

\subsection{Components}
Light waves have both vertical and horizontal components. The horizontal component is called the "electric field oscillation" and the vertical component is called the "magnetic field oscillation".\\\\
\includegraphics[scale=0.45]{images/polarization} \\

\subsection{Lens}
\noindent As a light wave travels through a polarized lens, either the vertical or horizontal component is eliminated, leaving only the component of the polarized lens.
For example, if you put together vertically polarized lens and a horizontally polarized lens in their corresponding directions, the light will not be able to move through the lens.
\\\\If you rotate one of the lens by 90$\degree$, the light will be able to pass through both of the lens since the polarized component is now the same as the other lens.

\subsection{3D Glasses}
When recording a 3D film, one camera records only the x components, and another camera records only the y components. Both cameras record at slightly different positions.
When you put on the 3D Glasses (which have polarized lenses), each eye sees a different image. Your brain then tries to put together the image which creates the 3D effect.\\\\
\includegraphics[scale=0.35]{images/3d_glasses} \\

\subsection{Circular Polarization}
Circularly polarized waves radiate energy in a horizontal plane, vertical plane and any other possible plane. Circularly polarized lenses grant us access to all the components of a wave instead of just one if using a linearly polarized lens.\\\\
\includegraphics[scale=0.5]{images/circular_polarization}\\

% Double Slit Equation 1
\section{Double Slit Solve Example 1}
Monochromatic light falls on two slits 0.024 mm apart. The fringes on a screen 3.00 m away are 6.7 cm apart. What is the wavelength of light?
\subsection*{Givens}
\begin{itemize}
    \item $d = 2.4 \times 20^{-5}\;m$
    \item $L = 3.00\;m $
    \item $\Delta x = 6.7 \times 10^{-2}\;m$
\end{itemize}\leavevmode
\subsection*{Solve}
To solve for the wavelength of the light, we can use the equation $\lambda = \left(\frac{(\Delta x)(d)}{L}\right)$\\ Our result should be in meters, but if converted, the result should be in nanometers.\\
\begin{align*}
     & \lambda = \left(\frac{(\Delta x)(d)}{L}\right)      = \left(\frac{(6.7 \times 10^{-2})(2.4 \times 10^{-5})}{3.00}\right) \approx 5.20 \times 10^{-7}\;m
\end{align*}\leavevmode\\


% Double Slit Equation 2
\section{Double Slit Solve Example 2}
A parallel beam of 700 nm light falls on two small slits $6.0 \times 10^{-2}$ mm apart. How wide would a pattern of eight bright fringes be on a screen 3.0 m away?
\subsection*{Givens}
\begin{itemize}
    \item $d = 6.0 \times 10^{-5}\;m$
    \item $\lambda = 7.0 \times 10^{-7}\;m$
    \item $L = 3.0\;m$
\end{itemize}\leavevmode
\subsection*{Solve}
To solve for the width of the pattern of eight bright fringes, we must first solve for the distance between two ($\Delta x$). After solving for $\Delta x$ we can multiply it's value by eight to get the final distance.\\
\begin{align*}
     & \Delta x = \frac{(\lambda)(L)}{d} = \left(\frac{(7.0 \times 10^{-7})(3.0)}{6.0 \times 10^{-5}}\right) \approx 3.5 \times 10^{-2}\;m \\\\
     & \therefore\;\;(8)\Delta x = (8)(3.5 \times 10^{-2}) \approx 2.8 \times 10^{-1}
\end{align*}\leavevmode\\

\vspace{3cm}

% Single Slit Equations
\section{Single Slit Solve Example 1}
How wide is the central diffraction peak on a screen 2.50 m behind a 0.0212 mm wide slit illuminated by 550 nm light?
\subsection*{Givens}
\begin{itemize}
    \item $L = 2.50\;m$
    \item $d = 2.12 \times 10^{-5}\;m$
    \item $\lambda = 5.5 \times 10^{-7}\;m$
\end{itemize}\leavevmode
\subsection*{Solve}
To solve for the width of the central maxima we use a formula very similar to the one used in the double slit calculations: $\Delta y = \left(\frac{(\lambda)(L)}{d}\right)$ Where $\Delta y$ is the width of the central maxima.\\
\begin{align*}
     & \therefore\;\;\Delta y = \left(\frac{(\lambda)(L)}{d}\right) = \left(\frac{(5.5 \times 10^{-7})(2.50)}{2.12 \times 10^{-5}}\right) \approx 6.5 \times 10^{-2}\;m
\end{align*}\leavevmode\\

% Single Slit Equation 2
\section{Single Slit Solve Example 2}
How wide is a slit if it diffracts 690 nm light so that its central peak is 3.0 cm wide on a screen 2.80 m away?
\subsection*{Givens}
\begin{itemize}
    \item $L = 2.80\;m$
    \item $ \Delta y = 3.0 \times 10^{-2}\;m$
    \item $\lambda = 6.9 \times 10^{-7}\;m$
\end{itemize}\leavevmode
\subsection*{Solve}
To solve for the width of the single slit we can rearrange the formula from the solve above. $\Delta y = \left(\frac{(\lambda)(L)}{d}\right)$ $\to$ $d = \left(\frac{(\lambda)(L)}{\Delta y}\right)$.\\
\begin{align*}
     & \therefore\;\;d = \left(\frac{(\lambda)(L)}{\Delta y}\right) = \left(\frac{(6.9 \times 10^{-7})(2.80)}{3.0 \times 10^{-2}}\right) \approx 6.44 \times 10^{-5}\;m
\end{align*}\leavevmode\\

\end{document}